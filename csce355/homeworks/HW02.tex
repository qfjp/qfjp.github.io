\documentclass[11pt]{article}
\usepackage{ifpdf}
\makeatletter
\edef\texforht{TT\noexpand\fi
  \@ifpackageloaded{tex4ht}
    {\noexpand\iftrue}
    {\noexpand\iffalse}}
\makeatother
\ifpdf
  \usepackage[framemethod=TikZ]{mdframed}
\else
  \usepackage{framed}
  \def\pgfsysdriver{pgfsys-tex4ht-alt.def}
\fi
\usepackage{amsfonts,amssymb,amsmath,amsthm,epsfig,mathtools,bm}
\usepackage{tikz, float}
\usepackage{parskip}
\usepackage{advdate}
\usepackage[linesnumbered,ruled]{algorithm2e}
\usepackage{enumitem}
\usepackage{booktabs}


\usetikzlibrary{shapes.geometric, arrows, automata, positioning, fit}
\tikzset {
    child/.style = {isosceles triangle, draw, shape border rotate = 90, anchor=north}
}

\newcommand{\nums}{\mathbb{N}}
\newcommand{\ints}{\mathbb{Z}}
\newcommand{\rats}{\mathbb{Q}}
\newcommand{\reals}{\mathbb{R}}
\newcommand{\complexes}{\mathbb{C}}
\newcommand{\cpxs}{\mathbb{C}}
\newcommand{\floor}[1]{{\lfloor{#1}\rfloor}}
\newcommand{\ceiling}[1]{{\lceil{#1}\rceil}}
\newcommand{\tuple}[1]{{\langle{#1}\rangle}}
\newcommand{\set}[1]{{\left\{{#1}\right\}}}
\newcommand{\bigfloor}[1]{{\left\lfloor{#1}\right\rfloor}}
\newcommand{\bigceiling}[1]{{\left\lceil{#1}\right\rceil}}
\newcommand{\bigtuple}[1]{{\left\langle{#1}\right\rangle}}
\newcommand{\card}[1]{{\|{#1}\|}}
\newcommand{\bigcard}[1]{{\left\|{#1}\right\|}}
\newcommand{\st}{\mathrel{:}}
\newcommand{\symdiff}{\mathop{\triangle}}
\newcommand{\emptystr}{\varepsilon}
\newcommand{\eclose}{\textsc{eclose}}
\newcommand{\tb}[1]{\textbf{#1}}
\newcommand{\merge}[2]{{{#1}\mathop{\textit{merge}}{#2}}}
\newcommand{\blank}{\text{\textvisiblespace}}
\newcommand{\Tau}{T}
\newcommand{\myand}{\mathrel{\wedge}}
\newcommand{\myor}{\mathrel{\vee}}
\newcommand{\mynot}{\neg}
\newcommand{\bs}[1]{\boldsymbol{#1}}
\newcommand{\cC}{\mathcal{C}}
\newcommand{\mn}{\textup{min}}
\newcommand{\map}[3]{{{#1}:{#2}\rightarrow{#3}}}
\newcommand{\regexp}{\textup{REG}}
\newcommand{\push}[1]{{\textup{push}\;{#1}}}
\newcommand{\pop}{{\textup{pop}}}
\newcommand{\turnstile}{\vdash}
\newcommand{\Turnstile}[1]{\underset{#1}{\turnstile}}
\newcommand{\turnstar}{\overset{*}{\vdash}}
\newcommand{\Turnstar}[1]{\underset{#1}{\overset{*}{\vdash}}}
\newcommand{\derives}{\Rightarrow}
\newcommand{\Derives}[1]{\underset{#1}{\Rightarrow}}
\newcommand{\derivestar}{\overset{*}{\derives}}
\newcommand{\Derivestar}[1]{\underset{#1}{\overset{*}{\derives}}}
\newcommand{\height}{\textup{height}}
\newcommand{\tup}[1]{\langle{#1}\rangle}
\newcommand{\powset}[1]{\mathcal{P}\left({#1}\right)}

\theoremstyle{plain}
\newtheorem{theorem}{Theorem}[section]
\newtheorem{lemma}{Lemma}[section]
\newtheorem{corollary}{Corollary}[section]
\newtheorem{proposition}{Proposition}[section]
\newtheorem{fact}{Fact}[section]

\theoremstyle{definition}
\newtheorem{definition}{Definition}[section]
\newtheorem{exercise}[definition]{Exercise}
\newtheorem{claim}[definition]{Claim}
\newtheorem{example}{Example}[section]

\newenvironment{remark}{\paragraph{Remark.}}{}
\newenvironment{automaton}{%
\begin{tikzpicture}[->, >=stealth', shorten >=1pt, auto, node distance=2.8cm, semithick, on grid]
    \tikzset{every state/.style={minimum size=1.8pt}}
    \tikzset{every state/.style={inner sep=1.8pt}}
    \tikzset{accepting/.style={double distance=2pt}}
}{%
\end{tikzpicture}
}

\ifpdf
  \newenvironment{Framed}[1]{%
    \mdfsetup{
      frametitle={\colorbox{white}{\space #1 \space}},
      innertopmargin=10pt,
      frametitleaboveskip=-\ht\strutbox,
      frametitlealignment=\center,
    }
    \begin{mdframed}[roundcorner=10pt]
      \vspace{-5mm}
  }{
    \end{mdframed}
  }
\else
  \newenvironment{Framed}[1]{%
    \begin{framed}
  }{
    \end{framed}
  }
\fi
\newenvironment{soln}[0]{%
  \begin{Framed}{Solution}
}{
  \end{Framed}
}
\title{Assignment 2: Basic Set Operations}
\date{Due: \AdvanceDate[4]\today}

\begin{document}

\maketitle

\begin{Framed}{Instructions}
    Make sure to submit your homework on \textbf{SINGLE-SIDED 8.5$\times$11
    INCH PAPER}. Write your \textbf{NAME ON EVERY PAGE} and
    \textbf{DO NOT STAPLE OR FOLD} the sheets.

    In this and future homeworks, questions marked as optional will not be used for quizzes or final exam questions.
\end{Framed}
\begin{enumerate}
    \item
        Find a counterexample to the statement, ``All odd natural numbers are
        prime.''  What is the least counterexample (NOTE:\@ 1 is \textbf{NOT} prime)?
    \item
        Here are some standard definitions of set operations.  I'm using the
        symbol ``$:=$'' to mean ``equals by definition.''  For any sets
        $A$ and $B$,
        \begin{description}
            \item[Union:] $A\cup B := \{ x \st x\in A \myor x\in B \}$.
            \item[Intersection:] $A\cap B := \{ x \st x\in A \myand x\in B \}$.
            \item[Difference (Relative Complement):] $A-B := \{ x \st x\in A
                \myand x\notin B \}$.
            \item[Symmetric Difference:] $A\symdiff B = (A-B)\cup(B-A)$.
        \end{description}
        For each of the four operations above, redraw the Venn diagram below,
        shading the region corresponding to the result of the operation.
        \begin{figure}[H]
          \centering
          \begin{tikzpicture}
            \def\radius{2cm};

            \coordinate (ceni);
            \coordinate[xshift=1.25*\radius] (cenii);
            \draw (ceni) circle (\radius);
            \draw (cenii) circle (\radius);
            \node[yshift=.8*\radius] at (ceni) {$A$};
            \node[yshift=.8*\radius] at (cenii) {$B$};

            \draw ([xshift=-20pt, yshift=20pt] current bounding box.north west)
              rectangle ([xshift=20pt, yshift=-20pt] current bounding box.south east);
          \end{tikzpicture}
        \end{figure}
    \item
        Illustrate via a venn diagram the following identity for any sets $A$ and $B$.
        \[ A\symdiff B = (A\cup B) - (A\cap B). \]
    \item
        Prove the identity in the last problem directly for any sets $A$ and $B$.
    \item
        Let $A$ and $B$ be any sets.  Which of the following sets must
        be empty?  For each of those that need not be, give a specific,
        concrete example where it is not (your examples should be as small as possible).
        \begin{enumerate}
            \item
                $A\cup\emptyset$
            \item
                $A\cap\emptyset$
            \item
                $A-\emptyset$
            \item
                $A\times\emptyset$
            \item
                $A-B$
            \item
                $A-A$
            \item
                $(A-B)\cap(B-A)$
            \item
                $(A-B)-(B-A)$
            \item
                $A-(B-A)$
            \item
                $A-(A-B)$
            \item
                $A-(A-A)$
            \item
                $(A-B)\cap B$
        \end{enumerate}
    %\item
    %    Prove directly that if $A$, $B$, and $C$ are any sets, then
    %    \[ A\times (B\cup C) = (A\times B)\cup(A\times C). \]
    %    That is, Cartesian product distributes over union.
    %\item
    %    Prove directly that if $A$, $B$, and $C$ are any sets, then
    %    \[ A\times (B\cap C) = (A\times B)\cap(A\times C). \]
    %    That is, Cartesian product distributes over intersection.
    %\item
    %    How many functions are there mapping $\{1,2\}$ into $\{1,2,3\}$?  How about $\{1,2,3\}$ into $\{1,2\}$?
    %\item
    %    This question generalizes the previous one.  Let $A$ and $B$ be any finite sets, with $\card A = m$ and $\card B = n$, for some $m,n\in\nums$.  In terms of $m$ and $n$, how many functions are there mapping $A$ into $B$?  That is, how many functions $f$ are there with domain $A$ such that $f(a) \in B$ for all $a\in A$?  [Hint: For each $a\in A$, how many choices are there for $f(a)$?]  Note: For any sets $A$ and $B$, we usually let $B^A$ denote the set of all functions mapping $A$ into $B$.
    %\item
    %    Here we give some basic definitions from graph theory.  A \emph{graph} is a pair $G = (V,E)$, where $V$ is a finite set (whose elements are called the \emph{vertices} of $G$) and $E$ is some collection of two-element subsets of $V$ (called the \emph{edges} of $G$).  If $u$ and $v$ are vertices of $G$ and $e = \{u,v\}$ is an edge of $G$, then we must have $u\ne v$, and we say that $u$ and $v$ are \emph{adjacent}.  In this case, we also say that $e$ is the edge \emph{connecting} $u$ with $v$.  (Also note that $e = \{v,u\}$ as well, so the order of $u$ and $v$ in the edge does not matter.)  For every vertex $u$ of $G$, the \emph{degree} of $u$ is the number of vertices adjacent to $u$.  Notice that if $G$ has $n$ vertices, then the degree of each vertex of $G$ must be a natural number at most $n-1$.

    %    Prove that any graph with $n\ge 2$ vertices must have two distinct vertices with the same degree.  An equivalent formulation of this problem is: Given a room with $n\ge 2$ people, any pair of which might or might not shake hands, there must be two different people who shake the same number of hands.  [Hint: Consider two cases: (i) there exists a vertex of degree $n-1$ (equivalently, there is someone who shakes everybody else's hand); (ii) there is no such vertex.  Use the pigeonhole principle.]

    %\item (Optional)
    %    [See the previous question for graph theory definitions.]  A \emph{complete graph on $n$ vertices} is a graph with $n$ vertices where every two distinct vertices are adjacent (so all possible edges are present in the graph).
    %    \begin{enumerate}
    %        \item
    %            Start with a complete graph $G$ on $5$ vertices.  $G$ has 10 edges.  Assign a color--- either red or blue---to each edge of $G$ in such a way that no set of three vertices of $G$ have the three edges connecting them all the same color.  (That is, there are no ``triangles'' that are all red or all blue.)
    %        \item
    %            Show that no such assignment is possible for a complete graph on $6$ vertices.  [Hint: By strong pigeonhole principle, a vertex must have at least three edges of the same color touching it.]
    %    \end{enumerate}
        %\item
        %Consider the binary string \texttt{1001101}.  Which of the following
        %languages (``problems'') over the binary alphabet $\{0,1\}$ does it
        %belong to?  (If necessary, interpret $w$ as a natural number in binary.)
        %\begin{enumerate}
        %\item $\{ w \st \mbox{$w$ represents a multiple of $11$}\}$
        %\item $\{w \st \mbox{$|w|$ is odd}\}$
        %\item $\{w \st \mbox{$w$ is a palindrome}\}$
        %\end{enumerate}
        %\item
        %How many binary strings are there of length $3$?  List them all.
        %How many are there of length $4$?  How many are there of length $n$
        %for any natural number $n$?
        %\item
        %Let $x$ and $y$ be strings over some fixed alphabet $\Sigma$.  We say
        %that
        %\begin{itemize}
        %\item
        %$x$ is a \emph{prefix} of $y$ iff there is some string $w$ over
        %$\Sigma$ such that $xw=y$,
        %\item
        %$x$ is a \emph{suffix} of $y$ if there is some string $w$ over
        %  $\Sigma$ such that $wx=y$, and
        %\item
        %$x$ is a \emph{substring} of $y$ if there are strings $w_1,w_2$ over
        %  $\Sigma$ such that $w_1xw_2 = y$.
        %\end{itemize}
        %Which of the following statements are true for all strings $x$, $y$,
        %and $z$ over $\Sigma$?  For those that are true, give direct proofs.
        %For those that are false, give counterexamples.
        %(Assume that $\Sigma$ has at least two symbols, say, \texttt{0} and
        %\texttt{1}, and possibly others.  Recall that $\emptystr$ denotes the
        %empty string over any alphabet.)
        %\begin{enumerate}
        %\item
        %$\emptystr$ is a prefix of $x$.
        %\item
        %$x$ is a prefix of $x$.
        %\item
        %If $x$ is a prefix of $y$, then $y$ is a prefix of $x$.
        %\item
        %If $x$ is a prefix of $y$, then $y$ is a suffix of $x$.
        %\item
        %If $x$ is a prefix of $y$ and $y$ is a prefix of $x$, then $x=y$.
        %\item
        %If $x$ is a prefix of $y$ and $y$ is a prefix of $z$, then $x$ is a
        %prefix of $z$.
        %\item
        %$x$ has the same number of prefixes as suffixes.
        %\item
        %If $x$ and $y$ are both prefixes of $z$, then either $x$ is a prefix
        %of $y$ or $y$ is a prefix of $x$.
        %\item
        %$x$ is a substring of $y$ iff $x$ is prefix of some suffix of $y$.
        %\item
        %$x$ is a substring of $y$ iff $x$ is a suffix of some prefix of $y$.
        %\item
        %If $x$ and $y$ are both substrings of $z$, then either $x$ is a
        %substring of $y$ or $y$ is a substring of $x$.
        %\end{enumerate}

\end{enumerate}

\end{document}
