\documentclass[11pt]{article}
\usepackage{ifpdf}
\makeatletter
\edef\texforht{TT\noexpand\fi
  \@ifpackageloaded{tex4ht}
    {\noexpand\iftrue}
    {\noexpand\iffalse}}
\makeatother
\ifpdf
  \usepackage[framemethod=TikZ]{mdframed}
\else
  \usepackage{framed}
  \def\pgfsysdriver{pgfsys-tex4ht-alt.def}
\fi
\usepackage{amsfonts,amssymb,amsmath,amsthm,epsfig,mathtools,bm}
\usepackage{tikz, float}
\usepackage{parskip}
\usepackage{advdate}
\usepackage[linesnumbered,ruled]{algorithm2e}
\usepackage{enumitem}
\usepackage{booktabs}


\usetikzlibrary{shapes.geometric, arrows, automata, positioning, fit}
\tikzset {
    child/.style = {isosceles triangle, draw, shape border rotate = 90, anchor=north}
}

\newcommand{\nums}{\mathbb{N}}
\newcommand{\ints}{\mathbb{Z}}
\newcommand{\rats}{\mathbb{Q}}
\newcommand{\reals}{\mathbb{R}}
\newcommand{\complexes}{\mathbb{C}}
\newcommand{\cpxs}{\mathbb{C}}
\newcommand{\floor}[1]{{\lfloor{#1}\rfloor}}
\newcommand{\ceiling}[1]{{\lceil{#1}\rceil}}
\newcommand{\tuple}[1]{{\langle{#1}\rangle}}
\newcommand{\set}[1]{{\left\{{#1}\right\}}}
\newcommand{\bigfloor}[1]{{\left\lfloor{#1}\right\rfloor}}
\newcommand{\bigceiling}[1]{{\left\lceil{#1}\right\rceil}}
\newcommand{\bigtuple}[1]{{\left\langle{#1}\right\rangle}}
\newcommand{\card}[1]{{\|{#1}\|}}
\newcommand{\bigcard}[1]{{\left\|{#1}\right\|}}
\newcommand{\st}{\mathrel{:}}
\newcommand{\symdiff}{\mathop{\triangle}}
\newcommand{\emptystr}{\varepsilon}
\newcommand{\eclose}{\textsc{eclose}}
\newcommand{\tb}[1]{\textbf{#1}}
\newcommand{\merge}[2]{{{#1}\mathop{\textit{merge}}{#2}}}
\newcommand{\blank}{\text{\textvisiblespace}}
\newcommand{\Tau}{T}
\newcommand{\myand}{\mathrel{\wedge}}
\newcommand{\myor}{\mathrel{\vee}}
\newcommand{\mynot}{\neg}
\newcommand{\bs}[1]{\boldsymbol{#1}}
\newcommand{\cC}{\mathcal{C}}
\newcommand{\mn}{\textup{min}}
\newcommand{\map}[3]{{{#1}:{#2}\rightarrow{#3}}}
\newcommand{\regexp}{\textup{REG}}
\newcommand{\push}[1]{{\textup{push}\;{#1}}}
\newcommand{\pop}{{\textup{pop}}}
\newcommand{\turnstile}{\vdash}
\newcommand{\Turnstile}[1]{\underset{#1}{\turnstile}}
\newcommand{\turnstar}{\overset{*}{\vdash}}
\newcommand{\Turnstar}[1]{\underset{#1}{\overset{*}{\vdash}}}
\newcommand{\derives}{\Rightarrow}
\newcommand{\Derives}[1]{\underset{#1}{\Rightarrow}}
\newcommand{\derivestar}{\overset{*}{\derives}}
\newcommand{\Derivestar}[1]{\underset{#1}{\overset{*}{\derives}}}
\newcommand{\height}{\textup{height}}
\newcommand{\tup}[1]{\langle{#1}\rangle}
\newcommand{\powset}[1]{\mathcal{P}\left({#1}\right)}

\theoremstyle{plain}
\newtheorem{theorem}{Theorem}[section]
\newtheorem{lemma}{Lemma}[section]
\newtheorem{corollary}{Corollary}[section]
\newtheorem{proposition}{Proposition}[section]
\newtheorem{fact}{Fact}[section]

\theoremstyle{definition}
\newtheorem{definition}{Definition}[section]
\newtheorem{exercise}[definition]{Exercise}
\newtheorem{claim}[definition]{Claim}
\newtheorem{example}{Example}[section]

\newenvironment{remark}{\paragraph{Remark.}}{}
\newenvironment{automaton}{%
\begin{tikzpicture}[->, >=stealth', shorten >=1pt, auto, node distance=2.8cm, semithick, on grid]
    \tikzset{every state/.style={minimum size=1.8pt}}
    \tikzset{every state/.style={inner sep=1.8pt}}
    \tikzset{accepting/.style={double distance=2pt}}
}{%
\end{tikzpicture}
}

\ifpdf
  \newenvironment{Framed}[1]{%
    \mdfsetup{
      frametitle={\colorbox{white}{\space #1 \space}},
      innertopmargin=10pt,
      frametitleaboveskip=-\ht\strutbox,
      frametitlealignment=\center,
    }
    \begin{mdframed}[roundcorner=10pt]
      \vspace{-5mm}
  }{
    \end{mdframed}
  }
\else
  \newenvironment{Framed}[1]{%
    \begin{framed}
  }{
    \end{framed}
  }
\fi
\newenvironment{soln}[0]{%
  \begin{Framed}{Solution}
}{
  \end{Framed}
}

\title{Assignment 1: Basic Proofs}
\date{Due: Jan 18, 2016}

\begin{document}

\maketitle

\begin{Framed}{Instructions}
  Make sure to submit your homework on \textbf{SINGLE-SIDED
  8.5$\times$11 INCH PAPER}. Write your \textbf{NAME ON EVERY PAGE}
  and \textbf{DO NOT STAPLE OR FOLD} the sheets.

  In this and future homeworks, questions marked as optional will not
  be used for quizzes or final exam questions.
\end{Framed}

The point of this homework is to give you practice proving basic facts
in arithmetic (ones that you all know), but only using the barest of
assumptions while doing so.  This promotes economy of expression and
forces you to be very clear about the concepts involved.

Here are some common letters we use to represent number systems:
\begin{itemize}
  \item
    $\nums = \{0,1,2,\ldots\}$ is the set of \emph{natural numbers}.
    Notice that this set includes zero.
  \item
    $\ints = \{\ldots,-2,-1,0,1,2,\ldots\}$ is the set of
    \emph{integers}.
  \item
    $\rats = \{ a/b \mid a,b \in \ints \text{ and } b \ne 0
    \}$ is the set of \emph{rational numbers}.
  \item $\reals$ is the set of \emph{real numbers}.
\end{itemize}
We have $\nums \subseteq \ints \subseteq \rats \subseteq \reals$.
Note that we include zero as the first natural number.  Other people
start the natural numbers with one instead.

\begin{definition}\label{def:even}\rm
An integer $m$ is \emph{even} if $m = 2k$ for some integer $k$;
otherwise, $m$ is \emph{odd}.
\end{definition}

\begin{definition}\label{def:tree}\rm
    If $T_1, T_2, \ldots, T_k$ are trees, then we can form a new tree
    as follows:
    \begin{enumerate}
        \item
            Begin with a new node $N$, which is the root of the tree
        \item
            Add copies of all the trees $T_1, T_2, \ldots, T_k$
        \item
            Add edges from node $N$ to the roots of each of the trees
            $T_1, T_2, \dots, T_k$
    \end{enumerate}
\end{definition}
\begin{figure}[H]
  \centering
  \begin{tikzpicture}
      \node[circle, draw]{N}
          child[edge from parent path = {(\tikzparentnode) -- (\tikzchildnode.north)}] {node[child] (1) {$T_1$}}
          child[edge from parent path = {(\tikzparentnode) -- (\tikzchildnode.north)}] {node[child] (2) {$T_2$}}
          child[edge from parent path = {(\tikzparentnode) -- (\tikzchildnode.north)}] {node[child, xshift=1cm] (k) {$T_k$}};
        \path (2) -- (k) node [midway] {$\circ\;\circ\;\circ$};
  \end{tikzpicture}
  \caption{Inductive construction of a tree}
\end{figure}

\section*{Exercises}
\begin{enumerate}
\item
Using just Definition~\ref{def:even} above and elementary facts about
arithmetic, prove directly that for integers $a$ and $b$, if at least
one of them is even, then $ab$ is even.
\item
With the same assumptions as in the previous exercise, prove directly
that if $a$ and $b$ are even, then $a+b$ and $-a$ are even.
\item
With the same assumptions as in the previous exercise, prove by
\emph{contradiction} that if $a$ is even and $b$ is odd, then $a+b$ is
odd.  That is, assume that $a$ is even and $b$ is odd, but that $a+b$
is even; then derive a contradiction.  (The only fact you can use
about a number being odd is that it is not even.  For example,
\emph{don't} use the fact (not proven yet) that any odd number is of
the form $2k+1$ for some integer $k$.)
\item
Now using the fact that all odd integers are of the form $2k+1$ where
$k\in\ints$, prove directly that the sum of two odd integers is even.
\item
Prove the following facts hold for all natural numbers $n$ by using induction.
\begin{itemize}
  \item $\sum\limits_{i=1}^n i = \dfrac{n(n+1)}{2}$
  \item $\sum\limits_{i=1}^n i^3 = \frac{1}{4}(n^4 + 2n^3 + n^2) = \dfrac{n^2{(n+1)}^2}{4}$
 \end{itemize}
%\item
%Consider the inductive definition of a tree given in Section~1.4 of
%the text (Definition ~\ref{def:tree} above).  Using this definition, give an inductive definition of the
%\emph{depth} of a tree (i.e., the longest length of a simple path from the
%root to any node).
\item
What is the largest integer that cannot be expressed as a sum of
$4$'s and $7$'s?  That is, find the largest integer $n$ such that $n \ne
4k+7\ell$ for any integers $k,\ell \ge 0$.  Prove your answer.  (Your
proof should use induction.)
\item
The \emph{associative law} (for real multiplication) states that, for any $a,b,c\in \reals$, we have $(ab)c = a(bc)$.  Using just this law (repeatedly), prove that $((ab)c)d = a(b(cd))$ for any $a,b,c,d\in\reals$.  To avoid ambiguity, you should fully parenthesize all your expressions.  A string of equalities is a valid method of proof.
\end{enumerate}

\end{document}
