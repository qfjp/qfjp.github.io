\documentclass[11pt]{article}
\usepackage{ifpdf}
\makeatletter
\edef\texforht{TT\noexpand\fi
  \@ifpackageloaded{tex4ht}
    {\noexpand\iftrue}
    {\noexpand\iffalse}}
\makeatother
\ifpdf
  \usepackage[framemethod=TikZ]{mdframed}
\else
  \usepackage{framed}
  \def\pgfsysdriver{pgfsys-tex4ht-alt.def}
\fi
\usepackage{amsfonts,amssymb,amsmath,amsthm,epsfig,mathtools,bm}
\usepackage{tikz, float}
\usepackage{parskip}
\usepackage{advdate}
\usepackage[linesnumbered,ruled]{algorithm2e}
\usepackage{enumitem}
\usepackage{booktabs}


\usetikzlibrary{shapes.geometric, arrows, automata, positioning, fit}
\tikzset {
    child/.style = {isosceles triangle, draw, shape border rotate = 90, anchor=north}
}

\newcommand{\nums}{\mathbb{N}}
\newcommand{\ints}{\mathbb{Z}}
\newcommand{\rats}{\mathbb{Q}}
\newcommand{\reals}{\mathbb{R}}
\newcommand{\complexes}{\mathbb{C}}
\newcommand{\cpxs}{\mathbb{C}}
\newcommand{\floor}[1]{{\lfloor{#1}\rfloor}}
\newcommand{\ceiling}[1]{{\lceil{#1}\rceil}}
\newcommand{\tuple}[1]{{\langle{#1}\rangle}}
\newcommand{\set}[1]{{\left\{{#1}\right\}}}
\newcommand{\bigfloor}[1]{{\left\lfloor{#1}\right\rfloor}}
\newcommand{\bigceiling}[1]{{\left\lceil{#1}\right\rceil}}
\newcommand{\bigtuple}[1]{{\left\langle{#1}\right\rangle}}
\newcommand{\card}[1]{{\|{#1}\|}}
\newcommand{\bigcard}[1]{{\left\|{#1}\right\|}}
\newcommand{\st}{\mathrel{:}}
\newcommand{\symdiff}{\mathop{\triangle}}
\newcommand{\emptystr}{\varepsilon}
\newcommand{\eclose}{\textsc{eclose}}
\newcommand{\tb}[1]{\textbf{#1}}
\newcommand{\merge}[2]{{{#1}\mathop{\textit{merge}}{#2}}}
\newcommand{\blank}{\text{\textvisiblespace}}
\newcommand{\Tau}{T}
\newcommand{\myand}{\mathrel{\wedge}}
\newcommand{\myor}{\mathrel{\vee}}
\newcommand{\mynot}{\neg}
\newcommand{\bs}[1]{\boldsymbol{#1}}
\newcommand{\cC}{\mathcal{C}}
\newcommand{\mn}{\textup{min}}
\newcommand{\map}[3]{{{#1}:{#2}\rightarrow{#3}}}
\newcommand{\regexp}{\textup{REG}}
\newcommand{\push}[1]{{\textup{push}\;{#1}}}
\newcommand{\pop}{{\textup{pop}}}
\newcommand{\turnstile}{\vdash}
\newcommand{\Turnstile}[1]{\underset{#1}{\turnstile}}
\newcommand{\turnstar}{\overset{*}{\vdash}}
\newcommand{\Turnstar}[1]{\underset{#1}{\overset{*}{\vdash}}}
\newcommand{\derives}{\Rightarrow}
\newcommand{\Derives}[1]{\underset{#1}{\Rightarrow}}
\newcommand{\derivestar}{\overset{*}{\derives}}
\newcommand{\Derivestar}[1]{\underset{#1}{\overset{*}{\derives}}}
\newcommand{\height}{\textup{height}}
\newcommand{\tup}[1]{\langle{#1}\rangle}
\newcommand{\powset}[1]{\mathcal{P}\left({#1}\right)}

\theoremstyle{plain}
\newtheorem{theorem}{Theorem}[section]
\newtheorem{lemma}{Lemma}[section]
\newtheorem{corollary}{Corollary}[section]
\newtheorem{proposition}{Proposition}[section]
\newtheorem{fact}{Fact}[section]

\theoremstyle{definition}
\newtheorem{definition}{Definition}[section]
\newtheorem{exercise}[definition]{Exercise}
\newtheorem{claim}[definition]{Claim}
\newtheorem{example}{Example}[section]

\newenvironment{remark}{\paragraph{Remark.}}{}
\newenvironment{automaton}{%
\begin{tikzpicture}[->, >=stealth', shorten >=1pt, auto, node distance=2.8cm, semithick, on grid]
    \tikzset{every state/.style={minimum size=1.8pt}}
    \tikzset{every state/.style={inner sep=1.8pt}}
    \tikzset{accepting/.style={double distance=2pt}}
}{%
\end{tikzpicture}
}

\ifpdf
  \newenvironment{Framed}[1]{%
    \mdfsetup{
      frametitle={\colorbox{white}{\space #1 \space}},
      innertopmargin=10pt,
      frametitleaboveskip=-\ht\strutbox,
      frametitlealignment=\center,
    }
    \begin{mdframed}[roundcorner=10pt]
      \vspace{-5mm}
  }{
    \end{mdframed}
  }
\else
  \newenvironment{Framed}[1]{%
    \begin{framed}
  }{
    \end{framed}
  }
\fi
\newenvironment{soln}[0]{%
  \begin{Framed}{Solution}
}{
  \end{Framed}
}
\title{CSCE 355, Spring 2016, Assignment 9}
\date{Due: \AdvanceDate[7]\today}
\begin{document}

\maketitle

\begin{Framed}{Instructions}
  Make sure to submit your homework on \textbf{SINGLE-SIDED 8.5x11
  INCH PAPER}. Write your \textbf{NAME ON EVERY PAGE} and
  \textbf{DO NOT STAPLE OR FOLD} the sheets.

  In this and future homeworks, questions marked as optional will
  not be used for quizzes or final exam questions.
\end{Framed}

The Exercises are taken out of Hopcroft, with their difficulty
annotations preserved.

\begin{description}

  \item[Exercise~4.2.1:] Suppose $h$ is the homomorphism from the
    alphabet $\{0,1,2\}$ to the alphabet $\{a,b\}$ defined by $h(0) =
    a$; $h(1) = ab$, and $h(2) = ba$.
    \begin{enumerate}[label={\alph*)}]
      \item What is $h(0120)$?
      \item What is $h(21120)$?
      \item If $L$ is the language $L(\tb 0 \tb 1^*\tb 2)$, what is $h(L)$?
      \item If $L$ is the language $L(\tb 0 + \tb{12})$, what is $h(L)$?
      \item Suppose $L$ is the language $\{ababa\}$, that is, the language
        consisting of only the one string $ababa$.  What is $h^{-1}(L)$?
      \item If $L$ is the language $L(\tb a(\tb{ba})^*)$, what is
        $h^{-1}(L)$?
    \end{enumerate}

\item[(not in text):]
  Let $h$ be the homomorphism mapping strings in $\{a,b,c\}^*$ to strings in
  $\{0,1\}^*$ such that
  \begin{eqnarray*}
    h(a) & = & 011, \\
    h(b) & = & 0, \\
    h(c) & = & 110.
  \end{eqnarray*}
  \begin{enumerate}[label={\alph*)}]
    \item
      Find $h^{-1}(L)$, where $L = \{ 00110, 000111 \}$.
    \item
      Using the transformation described in class, give a regular expression
      for $h(M)$, where $M$ is the regular language denoted by $a^*(bc + ca)^*$.
    \item
      Using the construction described in class, construct a DFA for
      $h^{-1}(N)$, where $N$ is recognized by the following DFA:
      \begin{figure}[H]
        \centering
        \begin{automaton}
          \node[state, initial] (q0) {$q_0$};
          \node[state, accepting] (q1) [right = of q0] {$q_1$};
          \node[state, accepting] (q2) [below = of q0] {$q_2$};
          \node[state] (q3) [below = of q1] {$q_3$};

          \path (q0) edge node {$0$} (q1)
                     edge [bend right] node [left] {$1$} (q2);
          \path (q1) edge [bend right = 60] node [above] {$0$} (q0)
                     edge [] node [above] {$1$} (q2);
          \path (q2) edge [] node [right] {$0$} (q0)
                     edge [] node [above] {$1$} (q3);
          \path (q3) edge [] node [right] {$0$} (q1)
                     edge [loop right] node [right] {$1$} ();
        \end{automaton}
      \end{figure}
  \end{enumerate}


%\item[Exercise~4.2.5:] The operation of Exercise~4.2.3 is sometimes
%  viewed as a ``derivative,'' and $a\backslash L$ is written
%  $\frac{dL}{da}$.  These derivatives apply to regular expressions in
%  a manner similar to the way ordinary derivatives apply to arithmetic
%  expressions.  Thus, if $R$ is a regular expression, we shall use
%  $\frac{dR}{da}$ to mean the same as $\frac{dL}{da}$, if $L = L(R)$.
%\begin{enumerate}
%\renewcommand{\theenumi}{\alph{enumi})}
%\item Show that $\frac{d(R+S)}{da} = \frac{dR}{da} + \frac{dS}{da}$.
%\item Give the rule for the ``derivative'' of $RS$.  \emph{Hint}: You
%  need to consider two cases: if $L(R)$ does or does not contain
%  $\emptystr$.  This rule is not quite the same as the ``product
%  rule'' for ordinary derivatives, but is similar.
%\item Give the rule for the ``derivative'' of a closure, i.e.,
%  $\frac{d(R^*)}{da}$.
%\item Use the rules from (a)--(c) to find the ``derivatives'' of
%  regular expression $(\tb 0 + \tb 1)^*\tb{011}$ with respect to $0$
%  and $1$.
%\item Characterize those languages $L$ for which $\frac{dL}{d0} =
%  \emptyset$.
%\item Characterize those languages $L$ for which $\frac{dL}{d0} = L$.
%\end{enumerate}

\item[! Exercise 4.2.6: (optional)]
  Show that the regular languages are closed under the following
  operations:
  \begin{enumerate}[label={\alph*)}]
    \item $\textit{min}(L) = \{w\mid \mbox{$w$ is in $L$, but no proper
      prefix of $w$ is in $L$} \}$.
    \item $\textit{max}(L) = \{w\mid \mbox{$w$ is in $L$ and for no string $x$
      other than $\emptystr$ is $wx$ in $L$} \}$.
    \item $\textit{init}(L) = \{ w\mid \mbox{for some string $x$, $wx$ is in $L$}
      \}$.
  \end{enumerate}
  \emph{Hint}: Like Exercise~4.2.2, it is easiest to start with a DFA
  for $L$ and perform a construction to get the desired language.

\item[*!! Exercise~4.2.8 (optional):] Let $L$ be a language.  Define
  $\textit{half}(L)$ to be the set of first halves of strings in $L$,
  that is, $\{ w \mid$ for some $x$ such that $|x|=|w|$, we have $wx$
  in $L\}$.  For example, if $L = \{\emptystr,0010,011,010110\}$ then
  $\textit{half}(L) = \{\emptystr,00,010\}$.  Notice that odd-length
  strings do not contribute to $\textit{half}(L)$.  Prove that if $L$
  is a regular language, so is $\textit{half}(L)$.

\item[!! (not in text; optional):]  Let $L$ be a language.  Define
  $\sqrt L$ to be the set $\{ w\mid \mbox{$ww$ is in $L$} \}$.  Prove
  that if $L$ is regular, then so is $\sqrt L$.  \emph{Hint:} Given a
  DFA for $L$ with $n$ many states, one can build a DFA for
  $\sqrt L$ with $n^n$ many states.

  More generally, for $k\ge 1$, define $\sqrt[k]{L}$ to be the set $\{ w
  \mid \mbox{$w^k$ is in $L$} \}$.  Also define $\sqrt[*]{L}$ to be the
  set $\{ w \mid \mbox{$w$ is in $\sqrt[k]{L}$ for some $k\ge 1$} \}$.
  Similar constructions prove that if $L$ is regular, then all these
  other languages are regular.

\item[! Exercise~4.2.13:] We can use closure properties to help prove
  certain languages are not regular.  Start with the fact that the
  language
  \[ L_{0n1n} = \{0^n1^n \mid n\ge 0\} \]
  is not regular.  Prove the following languages not to be regular
  by transforming them, using operations known to preserve regularity,
  to $L_{0n1n}$:
  \begin{enumerate}[label={\alph*)}]
  \item $\{0^i1^j \mid i\ne j\}$.
  \item $\{0^n1^m2^{n-m} \mid n\ge m \ge 0\}$.
\end{enumerate}
\item[Exercise~4.2.14:] In Theorem~4.8, we described the ``product
  construction'' that took two DFAs and constructed one DFA whose
  languages is the intersection of the languages of the first two.
  \begin{enumerate}
    \renewcommand{\theenumi}{\alph{enumi})}
    \setcounter{enumi}{2}
  \item Show how to modify the product construction so the resulting DFA
    recognizes the difference of the languages of the two given DFAs.
  \item Show how to modify the product construction so the resulting DFA
    recognizes the union of the languages of the two given DFAs.
\end{enumerate}

\item[(not in text):] Let $L$ be a language.  Define
  $\textit{subseq}(L)$ to be the set of all strings obtained by removing
  zero or more symbols (anywhere) from a string in $L$ and closing up
  the gaps.  For example, if $L = \{aabc,cab\}$, then
  \[ \textit{subseq}(L) =
  \{\emptystr,a,b,c,aa,ab,ac,bc,aab,aac,abc,ca,cb,cab\}. \]
  Prove that if $L$ is regular, then so is $\textit{subseq}(L)$.
  ($\textit{subseq}$ is short for ``subsequence.''  Actually, it is
  known that if $L$ is \emph{any language whatsoever}, then
$\textit{subseq}(L)$ is regular.)
\emph{Hint}: You can do this either by transforming regular
expressions or by transforming $\emptystr$-NFAs.

\item[! (not in the text; optional)]
  Let $x$ and $y$ be any two strings.  A \emph{merge} of $x$
  and $y$ is any string obtained by merging the symbols of $x$ with
  those of $y$ in some arbitrary way, maintaining the order of the
  symbols from each string.  More exactly, if $|x|=m$ and $|y|=n$,
  then a string $z = z_1\cdots z_k$ is a merge of $x$ and $y$ if and
  only if
  \begin{itemize}
    \item $k = m+n$,
    \item there exist $1\le i_1<i_2<\cdots<i_m\le k$ such that $x =
      z_{i_1}z_{i_2}\cdots z_{i_m}$,
    \item there exist $1\le j_1<j_2<\cdots<j_n\le k$ such that $y =
      z_{j_1}z_{j_2}\cdots z_{j_n}$, and
    \item $\{i_1,\ldots,i_m\} \cap \{j_1,\ldots,j_n\} = \emptyset$.
  \end{itemize}
  For example, there are five different merges of the strings $ab$ and
  $bc$:
  \[ abbc \;\; abcb \;\; babc \;\; bacb \;\; bcab \]

  Let $A$ and $B$ be any languages over the same input alphabet
  $\Sigma$.  Define
  \begin{align*}
    \merge{A}{B} \coloneqq \{ z\in\Sigma^* \mid \mbox{$z$ is a merge of some $x\in A$
and some $y\in B$} \}.
\end{align*}
  Show that if $A$ and $B$ are regular, then so is $\merge{A}{B}$.
  \emph{Hint}: Given a DFA for $A$ with $r$ many states and an DFA for
  $B$ with $s$ many states, you can construct an \emph{NFA} for $\merge{A}{B}$
  with $rs$ many states.

\end{description}

\end{document}
